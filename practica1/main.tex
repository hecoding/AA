\documentclass{article}
\usepackage[utf8]{inputenc}
\usepackage[spanish]{babel}
\usepackage{graphicx}
\graphicspath{ {images/} }

\usepackage{listings}
\lstdefinestyle{code}{
language=Octave,
frame=single,
breakatwhitespace=true,
breaklines=true,
basicstyle=\small\ttfamily,
tabsize=4,
numbers=left,
numberstyle=\tiny,
columns=fullflexible
}
\lstdefinestyle{snippet}{
language=Octave,
breakatwhitespace=true,
breaklines=true,
basicstyle=\small\ttfamily,
}

%\addtolength{\textwidth}{1cm}

\title{Práctica 1}
\author{Héctor Laria Mantecón y Samuel Lapuente Jiménez}
\date{29 de octubre de 2015}

\begin{document}

\maketitle
\section{Introducción}
Dados unos ejemplos de entrenamiento, mediante varios métodos de regresión lineal llegamos a una función que estima el valor aproximado que deberían tomar las soluciones a unas entradas cualesquiera.

\section{Regresión lineal con una variable}
Utilizamos el método de descenso de gradiente para minimizar la función de coste $J(\theta)$ dada en el enunciado con una hipótesis $h(\theta)$ dada también. Para ello utilizamos diferentes scripts que se detallan en siguientes subsecciones.

Uniendo estos scripts utilizamos:
\lstinputlisting[style=code]{src/uno.m}
\pagebreak

\subsection{gradientDesc.m}
Esta función realiza el descenso de gradiente para encontrar la recta que mejor se ajusta a los ejemplos de entrenamiento.
\lstinputlisting[style=code]{src/gradientDesc.m}
\pagebreak % poner foto en cada hueco
\subsection{plotting.m}
Esta función imprime las gráficas XXXX.
\lstinputlisting[style=code]{src/plotting.m}

\subsection{cost.m}
Esta función calcula el coste, implementando las fórmulas $J(\theta)$ y $h(\theta)$.
\lstinputlisting[style=code]{src/cost.m}

\section{Regresión lineal con varias variables}
Ahora tenemos que aplicar el mismo método de regresión, teniendo varios atributos en los ejemplos de entrenamiento. El script principal es el siguiente:
\lstinputlisting[style=code]{src/dos.m}

\subsection{featureNormalize.m}
Lo primero que se pide es una función con esta cabecera
\begin{lstlisting}[style=snippet]
function[X\_norm, mu, sigma] = normalizaAtributo(X)
\end{lstlisting}
La cual es nuestra {\tt featureNormalize}:
\lstinputlisting[style=code]{src/featureNormalize.m}

\subsection{costMulti.m}
Hemos implementado la función $J$ como nos pedía en el cuadernillo.
\lstinputlisting[style=code]{src/costMulti.m}

\subsection{estudioMulti.m}
Y finalmente dibujamos las gráficas que nos piden, donde se muestra la evolución de la función de coste $J(\theta)$ a medida que avanza el descenso de gradiente, con distintos valores de tasa de aprendizaje.
\lstinputlisting[style=code]{src/estudioMulti.m}
% gráfica esa mierder

\section{Conclusión}
Como podemos ver, el descenso de gradiente funciona correctamente ya que como se aprecia en la gráfica XXX, la recta abarca el mayor número de puntos posibles ya que minimiza apropiadamente el coste de la gráfica YYY como se ve en las sucesivas iteraciones del algoritmo, representado en la gráfica ZZZ.
\vspace{3mm}

Para la segunda parte hemos visto que la normalización ha reducido considerablemente el tiempo de cómputo de la regresión.
La salida del script principal es:
\begin{lstlisting}[style=snippet]
resultadoGradiente = 293100
resultadoNormal = 283360
\end{lstlisting}
pero es peor que una solución analítica (dada por la ecuación normal {\tt pinv(X' * X) * X' * y}) porque el descenso de gradiente no deja de ser una aproximación.

Interpretando la gráfica que muestra el descenso de gradiente con diferentes $\alpha$, vemos que con uno mayor se acerca mucho más rápido al óptimo pero corremos el riesgo de {\it saltárnoslo} y nunca llegar a él, por eso debemos valorar entre coger un $\alpha$ suficientemente grande o pequeño.


\end{document}
