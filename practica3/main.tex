\documentclass{article}
\usepackage[utf8]{inputenc}
\usepackage[spanish]{babel}
\usepackage{subcaption}
\usepackage{graphicx}
\usepackage{pgffor}
\graphicspath{ {images/} }

\usepackage{listings}
\lstdefinestyle{code}{
language=Octave,
frame=single,
breakatwhitespace=true,
breaklines=true,
basicstyle=\small\ttfamily,
tabsize=4,
numbers=left,
numberstyle=\tiny,
columns=fullflexible
}
\lstdefinestyle{snippet}{
language=Octave,
breakatwhitespace=true,
breaklines=true,
basicstyle=\small\ttfamily,
}

\addtolength{\textwidth}{1cm}
\addtolength{\textheight}{0.75cm}

\title{Práctica 3}
\author{Héctor Laria Mantecón y Samuel Lapuente Jiménez}
\date{19 de noviembre de 2015}

\begin{document}

\maketitle

\section{Introducción}
Mediante el uso de regresión logística multi-clase además de redes neuronales, en esta práctica conseguimos decidir qué número escrito a mano está representado en la imagen de entrada.

\section{Regresión logística multi-clase}
Realizamos un entrenamiento por clase, es decir, construimos un modelo regresivo por cada clase que haya en los ejemplos dados utilizando la salida correcta para cada clase.

\subsection{Vectorización de la regresión logística}
Para un cálculo eficiente de la regresión, se computa el coste y el gradiente vectorialmente de la siguiente forma:
\lstinputlisting[style=code]{src/lrCostFunction.m}

\pagebreak
Cuya función se usa en el siguiente código para entrenar un clasificador por cada clase:
\lstinputlisting[style=code]{src/oneVsAll.m}

Finalmente el programa principal que utilizamos en este apartado es:
\lstinputlisting[style=code]{src/uno.m}
El cual nos da una salida:
\begin{lstlisting}
% of model hits:
 94.880
\end{lstlisting}


\section{Redes Neuronales}
En esta parte de la práctica, valiéndonos de una red neuronal ya entrenada, realizamos solamente la propagación hacia delante para obtener el valor de la hipótesis.
La cual está implementada de este modo:
\lstinputlisting[style=code]{src/forwardProp.m}

Y hacemos uso de ella de esta manera
\lstinputlisting[style=code]{src/dos.m}
para conseguir una salida:
\begin{lstlisting}
% of model hits:
 97.520
\end{lstlisting}

\section{Conclusión}
Viendo las salidas de los dos scripts principales, podemos afirmar que la red neuronal es notablemente más eficaz en comparación con los modelos individuales de la regresión multi-clase.

\end{document}